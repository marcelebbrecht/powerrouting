% titelseite.tex
% Deckblatt der Arbeit
%\begin{titlepage}
%\sffamily
%\vspace*{3cm}
%\begin{center}
%\large{\textbf{Energieeffizientes Routing in Ad-Hoc-Netzen}}\\
%\vspace*{0.25cm}
%\small{Eine simulationsbasierte Analyse von OLSR und AODV}
%\end{center}
%\end{titlepage}
%\thispagestyle{empty}
\sffamily
\thispagestyle{empty}
\section*{Vorwort}

Sie fragen sich jetzt vielleicht: \glqq Ein Vorwort bei einer Bachelorarbeit?\grqq. Ja, etwas ungewöhnlich. In der ursprünglichen Fassung war es auch nicht enthalten. Ich entschloss mich allerdings, mir ein Exemplar in Buchfassung zu drucken und wollte dem Ganzen einen professionellen Anstrich geben. Warum eigentlich? Ganz einfach: Ich halte diese Arbeit für wichtig. Im Moment des Verfassens dieser Zeilen befindet sich die Arbeit in Prüfung und ich hoffe sehr, dass es als positiv empfunden wird. Natürlich ist auch die Frage berechtigt, warum ich es überhaupt geschrieben habe. Nun, wie jeder Student muss ich eine Abschlussarbeit schreiben. So kam es zum Schreiben. Wie kam es aber zu dem Thema? Das ist, zumindest für mich, ganz einfach: Ich bin Netzwerk-Freak. Seit dem ich denken kann ... oder sagen wir, seit dem ich mit Computern umgehe, immerhin fast 25 Jahre, empfinde ich großen Spaß daran, diese kommunizieren zu lassen. Klar, am Anfang war das eher dadurch begründet, dass auf den LAN-Parties jemand benötigt wurde, der mehr als 2 Rechner vernetzen und solch ein Netz sauber betreiben kann, dennoch war es der Eintritt in meine Leidenschaft Netzwerk. Bis heute bin ich beruflich in diesem Bereich tätig, daher war ich sehr dankbar, dass ich in diesem Bereich meine Arbeit schreiben durfte.\newline

Dies wäre allerdings nicht ohne die Arbeit und Unterstützung vieler Menschen möglich gewesen, daher möchte ich mich ganz herzlich bei den Menschen, die Omnet++ und die Frameworks geschaffen haben, den Teilnehmern der Mailingliste die bereitwillig Fragen beantworten, aber auch bei meinen Freunden und meiner Frau für jegliche Unterstützung bedanken. Und nun viel Spaß beim lesen.\newline

\noindent Marcel Ebbrecht, 01.04.2018

\thispagestyle{empty}