% schluss.tex
% Kapitel 6: Fazit

\chapter{Fazit}
\label{chapter:fazit}

Es mag der Eindruck entstehen, dass die Auswertung ein wenig kurz geraten ist. Vielleicht liegt das aber auch daran, das es relativ gut funktioniert. Das gesteckte Ziel, eine faire Verteilung des Verkehrs zu erreichen, wurde, zumindest in den Kernexperimenten, erreicht. Die Funktion ist zudem in gemischten Netzen gegeben. Die Anpassung der Implementierung ist, sowohl im Simulator, als auch für echte Systeme relativ simpel gehalten, damit ein Test unter realen Bedingungen angebracht erscheint. Der Code beider Protokolle ist für den Linux-Kernel frei verfügbar \footnote{https://github.com/erimatnor/aodv-uu} \footnote{https://wiki.freifunk.net/Glossar\#OLSR}. Es stellt sich allerdings die Frage ob es sinnig ist diese beiden sehr alten Protokolle weiter zu entwickeln. Stattdessen erscheint es als sinnvoll diese Anpassungen bei moderneren Verfahren wie \textit{DANE} oder anderen auf den Stromverbrauch optimierten Protokollen zu testen. Wenn ein stromsparendes Verfahren zusätzlich noch den Verkehr fair verteilt, dann kann diese Technologie bei dem Aufbau moderner Ad-Hoc-Infrastrukturen durchaus zu höherer Akzeptanz für den Einsatz auf mobilen Geräten führen. Ich halte es persönlich für sehr realistisch, dass \gls{mesh} in kommenden Mobilfunkgenerationen eine erhebliche Rolle spielen wird. Nicht zuletzt weil es den Providern einen wirtschaftlichen Vorteil bringt. \newline